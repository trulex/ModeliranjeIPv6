\documentclass[11pt,a4paper,slovene]{myarticle}

%Uporabljeni paketi
\usepackage[slovene]{babel}
\usepackage[utf8]{inputenc}
\usepackage{lmodern}
\usepackage[T1]{fontenc}
\usepackage{fancyhdr}
\usepackage{caption}
\captionsetup{font={default,footnotesize}, labelfont=bf, format=hang,indention=.0cm}
\usepackage{graphicx,epsfig}
\usepackage{amsmath}
\usepackage{multirow}
\usepackage{color}
\usepackage{url}
\usepackage{makeidx}
\usepackage[official]{eurosym}
\usepackage{float}
\usepackage{hyperref}
\hypersetup{
   bookmarksnumbered=true,
   urlbordercolor={0 1 0},
   linkbordercolor={1 1 1},
   unicode=true,
   pdftitle={ Modeliranje Računalniških Omrežij },
   pdfauthor={Asistent},
   pdfdisplaydoctitle=true,
   pdftoolbar=true,
   pdfmenubar=true,
   pdfstartview=X Y Z
}

\urlstyle{same}

\setlength{\parskip}{12pt}
\setlength\parindent{0pt}
\setlength\unitlength{1mm}

\begin{document}
\label{naslov}
\pdfbookmark[1]{Naslov}{naslov}
\thispagestyle{empty}

\begin{center}
\begin{Large}
Modeliranje računalniških omrežij\\
Študijsko leto 2013/2014\\
\end{Large}

\vspace*{4cm}
\begin{LARGE}
\textbf{Modeliranje IPv6 omrežij\\}
\end{LARGE}
\vspace*{0.5cm}

\begin{Large}
1. delno poročilo velike seminarske naloge\\

\vspace*{4cm}

Nihad Kerić, 63090347\\
Miha Novak, 63100268\\
Gregor Bahor, 63090049\\
Darko Janković, 63100176\\

\vspace*{5cm}
Ljubljana, \today
\end{Large}
\end{center}

\pagebreak
\setcounter{page}{1}
\pagenumbering{arabic}


\label{Kazalo}
\pdfbookmark[1]{Kazalo}{Kazalo}
\tableofcontents
\thispagestyle{empty}
\pagebreak

\section{Naloga}
Modelirajte bolj kompleksne primere IPv6 omrežja s pomočjo INET ogrodja v orodju OMNeT++. 

\section{Opis 3 zgledov za modeliranje IPv6 omrežij v INET ogrodju.}
\subsection{Pv6NClients}
V datoteki NclientsEth.ned oz. NclientsPPP.ned imamo skonfigurirano omrežje s tremi IPv6 usmerjevalniki ter n IPv6 odjemalcev(komuniciranje preko aplikacije TelnetApp). Ipv6 odjemalci so strežnik in n klienti, kar je prikazano na sliki spodaj. Stevilo n-klientov v našem testnem primeru se doloci v konfiguraciji :[General]*.n=10, lahko tudi spremenimo čas izvajanja simulacije v nasem primeru 'sim-time-limit=168h' v datoteki omnetpp.ini. Vsi klienti so vezani na en usmerjevalnik r1. Med klienti in strežnikom so med seboj zaporedno vezani trije usmerjevalniki, strežnik je vezan na usmerjevalnik r3. Ob zagonu simulacije se vzpostavi stanje omrežja, nato pa se prične seja med strežnikom in klienti. Seje se izmenjujemo med razlicnimi klienti. Pri testiranju različno velikem številu klientov n=2,10,100,200 in pri simulacijskem času 168h ni prišlo do napak.Simulacija NClientsPPP je identična po zgradbi omrežja NclientsEth.
Razlika med omrežjema se pojavi v načinu povezave fiberline ali ethernetline in stem se spremeni cas potovanja paketov in propustnost kanalov.
Simulacija NclientsEth ima definirano hitrost prenosa podatkov datarate=100Mbps in je počasnejša od NClientsPPP, katera ima hitrost prenosa podatkov datarate=1Gbps.

\subsection{IPv6Bulk}
Omrežje sestavljajo strežnik, usmerjevalnik in trije odjemalci. Strežnik in vsi trije odjemalci so povezani z usmerjevalnikom, obstaja pa tudi direktna povezava med strežnikom in enimi izmed odjemalcem. Vse povezave so tipa in/out, hitrost prenosa podatkov po kanalu pa je 10Mbps z zakasnitvijo 0.1us. Ves promet v omrežju usmerja usmerjevalnik, ki usmerja tudi promet med strežnikom in odjemalcem. 
\\
Pred zagonom simulacije lahko izbiramo med različnimi implementacijami TCP
(Transmission Control Protocol):
\begin{itemize}
\item TCP, je protokol za nadzor prenosa podatkov, ki zagotavlja, da se informacije med prenosom ne izgubijo, ne spremenijo in da se informacije vnovič dostavijo, če je prišlo med prenosom do napake.
\item TCP\_lwIP, TCP lightweight IP, je široko uporabljen odprtokodni TCP/IP protokolni sklad oblikovan za uporabo v vgrajenih sistemih.
\item TCP\_NSC, implementacija TCP, ki je bila razvita v okviru NSC projekta (Network Simulation Cradle project)
\end{itemize} 
Opazujemo lahko izvajanje NDP (Neighbor Discovery Protocol) in TCP seje (trosmerno rokovanje, prenos podatkov).
\\
Paketi, ki se prenašajo pri NDP (Neighbor Discovery Protocol):
\begin{itemize}
\item RSpacket (Router Solicitation)
\item RApacket (Router Advertisement)
\item NSpacket (Neighbour Solicitation)
\item NApacket (Neighbour Advertisement)
\end{itemize}
Paketi, ki se prenašajo pri TCP seji
\begin{itemize}
\item SYN
\item SYN + ACK
\end{itemize}

\subsection{Nclients}
Nclient ima dve mreži :
1) NClientsEth.ned
2) NClientsPPP.ned

\subsubsection{NClientsEth}
Pri tistem omrežiju imamo komunikaciju med odjemalcem in strežnikom,ali pač z n odjemalcov pa enim strežnikom preko 3 usmerjevalnika. Kateri so povezani preko  ipv6 protokola, naslovi so razdelni na 8 naslovo.
V NClientsEth.ned fielu imam dve vrsti kanalov (channel):
- fiberline
-ethernetline
Kanali imata iste antribute z različnimi nastavitvi.
            fiberline (delay= 1us in datarate= 512 Mbps)
ethernetline (delay= 0.1us in datarate= 10 Mbps)
Usmerjevalniki komunicirajo preko tistih kanalov ,prvo uspostave povezavo pošiljanjem različnih paketov kot so: NSpacket  , RApacket, RSpacket  , SYN, SYN+ACK. Po vzpostavljenoj povezavi se začneju pošiljati paketov. Strežnik pošlje paket proti odjemalcu kateri pol odgovori z pošiljanjem paketa ACK.  Isto se zgodi pri pošiljanu paketov z strani odjemalca.

\subsubsection{NClientPPP}
Tudi v tem omrežiju imamo komunikaciju med n odjemalcev in strežnikom preko tri usmerjevalnika, ipv6 naslov je razdelen na 8 naslovo kateri so krajši od naslovo prve konfiguracije v temu se razlikujeta.
 V NClientsEth.ned fielu imam  kanal (channel):
            -fiberline z nastavitvemi: (delay= 1us in datarate= 512 Mbps)
 

\section{Podrobna analiza enega od zgledov}
Za analizo smo izbrali zgled \textit{demonetworketh}.
\begin{figure}[H]
\includegraphics[scale=0.5]{slike/demoNetworkEth.png}
\end{figure}
Omrežje je sestavljeno iz naslednjih gradnikov:
\begin{itemize}
\item configurator tipa FlatNetworkConfigurator6
\item r1 tipa Router6
\item r2 tipa Router6
\item cli[n] tipa StandardHost6
\item srv[n] tipa StandardHost6
\item linemonitor[n] tipa TCPDump
\end{itemize}

\textbf{FlatNetworkConfigurator6}\\
Konfigurira Ipv6 naslove in posredovalne tabele.

\textbf{Router6}\\
Predstavlja Ipv6 usmerjevalnik.

\textbf{StandardHost6}\\
Ipv6 gostitelj s TCP, SCTP in UDP plastmi in aplikacijami.

\textbf{TCPDump}\\
Pregledovanje vsebine paketov.

\section{Podroben opis razpoložljivih gradnikov INET ogrodja}
Opisali smo gradnike iz primera \textit{demonetworketh}.

\subsection{NetworkLayer6}
Omrežje vsebuje elemente, ki so ključnega pomena pri klientih, usmerjevalnikih in strežnikih.
Modul NetworkLayer6.ned predstavlja Ipv6 omrežno plast in  je sestavljen iz elementov:
-import inet.networklayer.ipv6.IPv6ErrorHandling;
-import inet.networklayer.ipv6.IPv6;
-import inet.networklayer.icmpv6.IPv6NeighbourDiscovery;
-import inet.networklayer.icmpv6.ICMPv6;
\\
SLIKA
\\

-Ipv6: modul setavlja klasifikacijski obrejkt (modul) IPv6Datagram, ki predstavlja glavo
paketa IPv6 protokola. Ko modul ipv6 pošlje paket višjemu nivoju (TCP ali UDP
protokol) po ISO/OSI omrežnem modelu ga opremi z izvornim in ponornim naslovom.
Opisani elementi povezujejo 3. omrežni (IPv6) in 4. transportni (TCP/UDP) nivo po
OSI/ISO omrežnem modelu.
\\
-IPv6ErrorHandling:Napake pridejo v obliki sporočila, modul se uporablja za beleženje napak na omrežnem nivoju.
\\
-NeighburDiscovery: modul se uporablja za izvajanje vseh naloge, povezanih z odkritje
sosedov in brez naslovno auto konfiguracijo. Neighbour discovery paketi so sami poslani in obdelani stem modulom. Ko Ipv6 prejme enega, posreduje paket naprej k IPv6 Neighbor Discovery.
\\
-Icmpv6: modul, ki služi za pošiljanje zahtev »echo request« na omrežnem
nivoju. Zahteva bo poslana na vrat pingIn z proloženo IPv6ControlInfo. Odgovor »echo reply« bo sprejet, ko sporočilo bo poslano skozi vrata pingOut.


\subsection{RoutingTable6}
Ta gradnik predstavlja IPv6 usmerjevalno tabelo. Vsak gostitelj ali usmerjevalnik v omrežju ima natanko en primerek tega gradnika. Je preprosti (simple) modul brez vrat in se aktivira s klicem funkcije. Ima funkcije za branje in posodabljanje tabele, ter za usmerjanje \textit{unicast} in \textit{multicast} prometa. Usmerjevalna tabela se inicializira na vrednosti, ki jih prebere iz datoteke. Med simulacijo se seveda iz nje bere ter se jo spreminja, skladno z uporabljenimi usmerjevalnimi protokoli.\\
Modul \textit{RoutingTable6.ned} ima 3 parametre:
\begin{itemize}
\item xml routingTable = default(xml("<routingTable/>")); // datoteka z začetno usmerjevalno tabelo v obliki XML
\item bool isRouter; // če je nastavljena na \textit{true}, je omogočeno posredovanje IP
\item bool forwardMulticast = default(false); // posredovanje multicast prometa
\end{itemize}

\subsection{InterfaceTable}
Gradnik predstavlja tabelo omrežnih vmesnikov. Omrežne vmesnike kot so PPPInterface in drugi, dinamično registrirajo ustrezni moduli. Tabela vsebuje lastnosti vmesnikov, ki se ne navezujejo na lastnosti protokolov. Specifične lastnosti protokolov, IPv4 in IPv6 se hranijo v gradniku RoutingTable oziroma v zgoraj opisanemu gradniku RoutingTable6. Modul InterfaceTable nima vrat in se aktivira s klici funkcij, ki se nahajajo v pripadajočem C++ razredu. Modul je opisan samo z enim parametrom displayAddresses, ki je tipa boolean in pove ali naj prikaže IP naslove na povezavah ali ne.

\subsection{Notification Board}
Z uporab Notification Boarda, lahko moduli sedaj med seboj obvestita o dogodki, kot je sprememba usmerjevalnih tabel, statusa vmesnika, konfiguracije vmesnik, brezžične predajanja itd.
Notification Board ima natanko en primerek v klientu ali strežniku model in deluje kot posrednik med modulom.
Lahko se pojavijo in moduli, ki so zainteresirani za učenje o tistih spremembe. Sej morajo biti imenovan kot "notificationBoard", da deluje pravilno.
Moduli lahko "naročite" na kategorij sprememb (npr. "usmerjevalne table spremeniti ali radijska postaja postala prazna"). Ko pride do take spremembe, ki ustrezajo
modul (npr. ~ RoutingTable ali modul fizični sloj) bo pustil  NotificationBoard.Obveščanja dogodki so združene v "kategoriji".
Vsaka kategorija je določena s celim številom (zdaj je to pripisano v izvorno kodo preko ENUM ).
Stranke, ki želijo prejemati obvestila je treba izvajati (podrazred od) v INotifiable vmesnik, dobimo kazalec na NotificationBoard, in se naročite na kategorije jih zanima 
s klicom subscribe() metodo v NotificationBoard. Obvestila bodo dostavljeni na receiveChangeNotification() metodo stranke (re definirani od INotifiable).



%\section{Zaključek}
%Ali ste izpolnili cilje in možne nadaljne nadgradnje. Pri %samem opisu rešitve se običajno sklicujemo na reference, npr. %\cite{omnetpp} in \cite{cisco}. 

\pagebreak
\bibliographystyle{plain}
\bibliography{references}

\end{document}














